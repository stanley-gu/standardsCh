\documentclass[12pt]{article}
\usepackage{html}
\usepackage{graphicx}
\usepackage{palatino}
\usepackage{setspace}
\usepackage{amsmath}
\usepackage{boxedminipage}
\usepackage{colortbl}
\usepackage{color}
\usepackage{chicago}
\onehalfspacing


\newcommand{\bgamma}{\mbox{\boldmath $\Gamma$}}
\newcommand{\bL}{\mbox{\boldmath $L$}}
\newcommand{\bT}{\mbox{\boldmath $T$}}
\newcommand{\bI}{\mbox{\boldmath $I$}}
\newcommand{\bM}{\mbox{\boldmath $M$}}
\newcommand{\bm}{\mbox{\boldmath $m$}}
\newcommand{\bN}{\mbox{\boldmath $N$}}
\newcommand{\bE}{\mbox{\boldmath $E$}}
\newcommand{\bA}{\mbox{\boldmath $A$}}
\newcommand{\bB}{\mbox{\boldmath $B$}}
\newcommand{\bK}{\mbox{\boldmath $K$}}
\newcommand{\bP}{\mbox{\boldmath $P$}}
\newcommand{\bx}{\mbox{\boldmath $x$}}
\newcommand{\bU}{\mbox{\boldmath $U$}}
\newcommand{\bV}{\mbox{\boldmath $V$}}
\newcommand{\bZero}{\mbox{\boldmath $0$}}
\newcommand{\bLo}{\mbox{\boldmath $L_0$}}
\newcommand{\bNo}{\mbox{\boldmath $N_0$}}
\newcommand{\bNr}{\mbox{\boldmath $N_R$}}
\newcommand{\bSi}{\mbox{\boldmath $S_i$}}
\newcommand{\bSd}{\mbox{\boldmath $S_d$}}
\newcommand{\bdSi}{\mbox{\boldmath $dS_i$}}
\newcommand{\bdSd}{\mbox{\boldmath $dS_d$}}
\newcommand{\bS}{\mbox{\boldmath $S$}}
\newcommand{\bdS}{\mbox{\boldmath $dS$}}
\newcommand{\bdt}{\mbox{\boldmath $dt$}}
\newcommand{\bdSdt}{\mbox{$\displaystyle \frac{\bdS}{\bdt}$}}
\newcommand{\bdSddt}{\mbox{$\displaystyle \frac{\bdS_d}{\bdt}$}}
\newcommand{\bdSidt}{\mbox{$\displaystyle \frac{\bdS_i}{\bdt}$}}
\newcommand{\bv}{\mbox{\boldmath $v$}}
\newcommand{\bp}{\mbox{\boldmath $p$}}

\parindent=0pt
\parskip=4pt

\begin{document}


\begin{figure} \label{table:mathTechniques}
\rule[-5pt]{14cm}{1pt}
\begin{description} 
\item[Boolean:] One of the simplest possible modelling techniques
is to represent a network using Boolean logic \cite{DeJong2002}.
This approach has been used to model gene networks.
%
\item[Ordinary differential equations (ODEs):] This is the
commonest and arguably most useful representation. Although based on
a continuum model, ODE models have proved to be excellent descriptions
of many biological systems. Another advantage to
using ODEs is the wide range of analytical and numerical methods that are available. The
analytical methods in particular provide a means to gain a deeper
insight into the workings of the model.
%
\item[Deterministic hybrid:] A deterministic hybrid model is one
which combines a continuous model (e.g ODE model) with discrete
events. These models are notoriously difficult to solve efficiently and
require carefully crafted numerical solvers. The events can occur
either in the state variables or parameters and can be
time dependent or independent. A simple example involves the
division of a cell into two daughter cells. This event can be
treated as a discrete event which occurs when the volume of the
cell reaches some preset value at which point the volume halves.
%
\item[Differential-algebraic equations (DAEs):] Sometimes a model
requires constraints on the variables during the solution of the
ODEs. Such a situation is often termed a DAE system. The simplest
constraints are mass conservation constraints, however these are
linear and can be handled efficiently and easily using simple
assignment equations (see equation \ref{eq:general}). DAE solvers need only
 be used when the constraints are nonlinear.
%
\item[Partial differential equations (PDEs):] Whereas simple ODEs
model well stirred reactors, PDEs can be used model heterogenous
spatial models. 
%
\item[Stochastic:] At the molecular level concentrations are
discrete, but as long as the concentrations levels are sufficiently high,
the continuous model is perfectly adequate. When concentrations
fall below approximately one hundred molecules in the volume considered (e.g. the cell or compartment) one has to consider using stochastic modelling. The great disadvantage in
this approach is that one looses almost all the analytical methods
that are available for continuous models, as a result stochastic
models are much more difficult to interpret. 
\end{description}
\rule[6pt]{14cm}{1pt}
\caption{ A non-exhaustive selection of mathematical techniques for modelling biological systems.}
\end{figure}

\end{document} 