
\subsection{Structural Analysis}

Large scale genomic studies are generating significant amounts of
data on the structure of cellular networks. This is in contrast to
kinetic data which is frequently absent, unreliable or
fragmentary. There is therefore a desire by many in the community
to investigate the potential rewards of analyzing the more readily
available topological data.

There are a number of areas where work has been carried out on
network structure. These include network connectivity, elementary
modes, flux balance and conservation analysis. Note that all these
approaches only look at the connectivity and do consider kinetics.

\subsubsection{Connectivity Studies}

Connectivity studies have centered around the search for patterns
in the way cellular networks are physically connected. It is
however not entirely clear whether the conclusions that have
emerged from this work are truly novel. One thing is certain, this
area has received a tremendous amount of coverage in both the
popular as well as academic press and has stimulated a great deal
of excitement and discussion.

One of the first difficulties that the connectionists have is how
to represent a biological network as a mathematical graph. In
constructing the mathematical graphs a number of simplifications
are made, these include irreversibility, that is edges in a graph
are unidirectional and common molecules such as ATP, Pi, NAD etc
do not contribute to the structure of the graph itself. With this
in mind, the connectionists then consider a number of network
metrics which are used to characterize the graphs. The most
important metrics include: degree distribution, clustering
distribution and path length. Theoretical studies of artificial
graphs indicate that these graph metrics can be associated with
particular graph patterns, the most important being random,
scale-free and hierarchical.

Random graphs are characterized by nodes that have roughly equal
numbers of edges and that nodes which deviate from this are
extremely rare. As a result, the degree distribution follows a
sharp poisson distribution and the clustering degree is constant.
The mean path length between nodes in a random graph is roughly
proportional to the logarithm of the network size which means that
all nodes are closely connected, so-called small-world property.

The main conclusion from these network studies is that biological
networks are not random since the network metrics do not follow
the expected pattern for a random graph. Instead, real networks
show two additional properties, scale-free and hierarchy.

Much has been made of the observation that real networks are
scale-free. What this means is that some nodes in the network are
highly connected where as most nodes are not. A visual inspection
of a pathway map will reveal this to be the case. What is not
clear however is what this means in practice. One suggestion is
that highly connected nodes, the so-called hubs, could be points
where disruption would lead to network failure. Clearly this kind
of disruption could only be made in protein networks where it is
possible through mutation to remove or affect a node's properties.
In metabolic networks, only the edges can be disrupted in this
way. However, what is missing from these arguments is the kinetic
factor. If is true that a network's configuration can affect a
network dynamic properties, however equally important are the
kinetic forces at work, including feedback loops and reaction
rates. These factors are ignored in these studies and one
therefore wonders how general the conclusion actually are.

The final conclusion that the connectionists make is that real
networks are not only scale-free and small-world but they are also
hierarchical. By this they mean that a network is made up of
highly clustered islands, with few links bridging the island.
Moreover they observe that islands of clusters form super
islands within a large structure.

The discoveries that the connectionists have made may appear at
one level to be quite obvious. Many of these ideas have been known
to the biological community for many years though they've never
been quantified. On the other hand, these studies have stimulated
a great deal of discussion and has brought to the attention of
biologists the need to take a more computational and theoretical
approach.


\subsubsection{Elementary Modes}

Elementary modes analysis is concerned with identifying subsets of
reactions whose net stoichiometry is balanced and within which all
internal reactions are also stoichiometrically balanced. Thus at
steady-state an elementary mode has no net consumption or
production of any internal substrate. Given this definition, an
elementary mode cannot be subdivided into further modes. An
elementary mode can be thought of as a minimal independent pathway
within a network of reactions that is capable of supporting a net
conversion. Any real flux distribution in the living cell is thus
a superposition of elementary modes. Elementary modes help to
clarify the notion of a pathway. Applications include maximizing
product yield in biosynthetic pathways, reconstruction and
consistency checks of metabolism from genome data, analysis of
enzyme deficiencies, and drug target identification in metabolic
networks.

\subsubsection{Flux Balance Analysis}

At steady state all consumption and production rates at every species
node are balanced. This is described by equating \ref{eq:system} to zero:

$$ \bN \bv = 0 $$ \label{eqn:steadyState}

This equation admits a multitude of solutions in $\bv$. These
solution represent all potential flow states in the network. By
the addition of constraints on the magnitude of the fluxes and
assuming that the solutions to (\ref{eqn:steadyState}) are subject
to an optimizing function expressed as a linear combination of the
fluxes contained in v, we can formulate equation
(\ref{eqn:steadyState}) as a linear programming problem:

\begin{eqnarray} \label{eqn:FluxBalance}
\bN \bv &=& 0 \\ \nonumber
0 \leq \alpha_i &\leq& v_i \leq \beta_i \\ \nonumber
\mbox{(minimize) maximize } \mbox{ } O &=& \sum c_i v_i \\ \nonumber
\end{eqnarray}

An example of a simple maximization function $O$ might be to
maximize the ATP turnover by maximizing the ATP hydrolysis flux.
This procedure will yield a set of fluxes that satisfies the
equation in (\ref{eqn:FluxBalance}).

\subsubsection{Conservation Analysis}

While the solutions to $\bN \bv = 0$ yields information on flux
balance, the solutions to

$$ \bN^t \bv = 0$$

yields solutions to mass balance constraints in the network. Such
constraints are invariable the result of conserved moieties. For
example in real networks, such moieties would include \{ATP, ADP,
ADP\}, \{NAD, NADH\} and so on. Detailed on how to compute the
conservation laws in arbitrary networks is given in Sauro and
Ingalls (2004).

Computing the conservation laws is useful for a number of reasons.
Most critically many of the numerical methods require a set of
independent systems equations (\ref{eq:general}), without which
they encounter severe numerical difficulties. This will problem
will be covered in more detail in the next section. In terms of
biological interest, the mass conservation laws give upper limits
on the concentrations that the participating species can reach.
For example, in the human parasite {\em Trypanosoma brucei} model
developed by Bakker and colleagues \cite{Bakker:1997}, it was
discovered that the metabolism of the organism was constrained by
four conservation laws, two relating to AMP another to NAD and a
third to phosphate. These constraints mean that attempts to
disrupt metabolite levels (by therapeutic drugs) in this organism
to levels that are harmful is limited. Conservation constraints
can impose hard limits to the extent to which a drug can influence
a metabolite level.


\subsection{Solving the System Equations}

The structural analysis of cellular networks can yield much
interesting information which can have practical as well and
academic implications. The next level of analysis is to
incorporate kinetics into the structure analysis. This involves
assigning rate laws to each reaction in the model and setting
values for the various parameters such as Km, Vmax and regulatory
constants. This is arguably, the most difficult aspect of
modelling, in many cases there will be little information on the
form of the rate laws, let along the values for the parameters.
For well studied pathways, such as a number of metabolic pathways,
the lack of information is less severe. For signaling pathways and
genetic networks, there is a significant lack of detailed kinetic
information. In the 60s and 70s it was almost routine for a
research group to record kinetic information on enzymes that they
had isolated. Unfortunately in the last twenty five years with the
rise of genomic research such studies have virtually vanished.
This means that all networks discovered since the early 80s,
mainly signalling and genetic have received less attention with
respect to kinetic measurements, hence the lack of information.

There are a number of techniques that one can use to alleviate the
lack of information, one is to apply model fitting techniques
which will be described later. A less known approach is to use
kinetic approximations combined with data fitting. There are a
number possible approximations one can use, these include simple
first order kinetics, power laws kinetics or more useful linlog
kinetics. One of the characteristics of biological processes is
that they saturate, this is an inevitable result of how biological
catalysis operates. Therefore what the approximate it should
attempt to replicate saturate at high levels of substrate. Linear
kinetic, although simple do not saturate, if it is known by some
means that the reaction operate always below saturation then a
first order approximation is quite suitable. The alternative is to use
power laws or linlog equations. Power law equations of the form:

$$ v = \alpha \sum S_i^{\varepsilon_i} $$

where the sum if over all effectors that could possible effect the
rate $v$. $\varepsilon_i$ is the elasticity with respect to
substrate $i$. This equation has the advantage that it has zero
rate at zero substrate, and that a intermediate substrate values
the function will begin to level off, however at high
concentration the function continues to rise and it's capacity to
saturate is quite limited. A better approach is to use a new
approximation, called linlog. The linlog approximation is given by
the equation:

\begin{equation} \label{eqn:linlog}
v = J^o \frac{e}{e^o} \left[ 1 + \varepsilon_{S_i}^{v} \frac{S_i}{S^o_i} +
  \varepsilon_{S_j}^v \frac{S_j}{S^o_j} + \ldots \right]
\end{equation}

where $\varepsilon$ are the elasticities for the different
effector. The linlog approximation is always computed relative to
some convenient operating point, the $x^o$ are the values at this
operating point. The advantage of this approximation is that the
function saturates quite well at high substrate con
concentrations. The only drawback is that the function does not go
through zero rate at low levels of substrate.

What is important however, is that both the power law and the
linlog are expressed in terms of elasticities which represent the
kinetic order of each effector:

$$ \varepsilon^v_S = \frac{\partial v}{\partial S} \frac{S}{v} $$

In many cases the kinetic order can be estimate from other
information such has the degree of irreversibility and saturation
of the reaction. Even if this information is not know, it is always
possible as a first approximation to sent the elasticities to
0.5. For activators the elasticities can be set to some value
greater than one but less than the hill coefficient of the
reaction. Inhibitors can be treated in the same way except the
values are negative.

Once the set of equation for the network has been constructed, it
is possible to carry out a number of different analyses. The most
obvious and what most researcher will do is to carry out a time
course simulation, that is to observe how the concentrations of
the various species evolve over time. The second most common
analysis to do is to compute the steady state of the model. Both
these techniques will be described next.

COR

\subsubsection{Time Course Solutions}

Plotting the time course solution of the model involves solving
the set of differential equations (\ref{eq:general}). There are
many standard techniques that are available to do this. It was
discovered early on in this field that cellular network models
tend to be what is called {\em stiff}. Stiffness arises when the
different parts of the network evolve in time at markedly
different rates. This means that the numerical method that
integrates the differential equations has to maintain a very small
time step in order to contain numerical error. This in turn makes
the simulation time much longer but more worrying it means that
the step size might have to be so small that rounding error
accumulate which causes the numerical method to fail. This problem
was very well appreciated by the early modelers and in the 1960s
there was a concerted effort to design numerical methods which
could deal with stiffness \cite{Ga77}. The result of
this research was the design of a whole family of what are known
as implicit integrators, the most celebrated being the one
developed by Gear. Since that time, various other variants have
been developed, most notable, LOSDA and CVODE \cite{Cohen1996}.
Both are highly recommended as the workhorse integrators for
modelling cellular networks. The latest release of CVODE (version
2.2) is particularly interesting because it has the capacity to
detect discrete change during a simulation. This is important if
the intention is to model deterministic-hybrid models.


METHOD OF SOLUTION - LSODE contains two variable-order, variable-
step (with interpolatory step-changing) integration methods. The
first is the implicit Adams or non-stiff method, of orders one
through twelve. The second is the backward differentiation or
stiff method (or BDF method, or Gear's method), of orders one
through five.

\subsubsection{Steady State Solutions}

NLEQ2

\subsection{Optimization}

Take sections from Vijay. Fitting Microarray data, Arkin

\subsection{Metabolic Control Analysis}

Metabolic Control Analysis, or more precisely, Cellular Control Analysis (CCA) is
a theory which describes how perturbations propagate through a network as the
network changes from one steady state to another steady state. Frank Bergmann in another part of this volume describes CCA in detail. Here it is suffice to say that CCA is an important
integrative tool for models.

\subsubsection{Frequency Control Analysis}

An extension to steady state control analysis is frequency control analysis (FCC), refs. This allows one to understand how a varying model parameter is transformed by the network. Because
FCC is a linearized analysis the only effect on the input signal is either to change
the amplitude or the phase shift.

$$ C_{S_i} (w) = \left( iw {\bf I} - {\bf N_r} \frac{\partial v}{\partial S} {\bf L} \right)^{-1} {\bf N_r} $$

\subsection{Bifurcation}

Usual, take section from grant proposal, list packages.

\subsection{Stochastic Simulation ?}

Brief
