\documentclass[12pt]{article}
\usepackage{html}
\usepackage{graphicx}
\usepackage{palatino}
\usepackage{setspace}
\usepackage{amsmath}
\usepackage{boxedminipage}
\usepackage{colortbl}
\usepackage{color}
\usepackage{chicago}
\onehalfspacing

\newcommand{\bgamma}{\mbox{\boldmath $\Gamma$}}
\newcommand{\bL}{\mbox{\boldmath $L$}}
\newcommand{\bT}{\mbox{\boldmath $T$}}
\newcommand{\bI}{\mbox{\boldmath $I$}}
\newcommand{\bM}{\mbox{\boldmath $M$}}
\newcommand{\bm}{\mbox{\boldmath $m$}}
\newcommand{\bN}{\mbox{\boldmath $N$}}
\newcommand{\bE}{\mbox{\boldmath $E$}}
\newcommand{\bA}{\mbox{\boldmath $A$}}
\newcommand{\bB}{\mbox{\boldmath $B$}}
\newcommand{\bK}{\mbox{\boldmath $K$}}
\newcommand{\bP}{\mbox{\boldmath $P$}}
\newcommand{\bx}{\mbox{\boldmath $x$}}
\newcommand{\bU}{\mbox{\boldmath $U$}}
\newcommand{\bV}{\mbox{\boldmath $V$}}
\newcommand{\bZero}{\mbox{\boldmath $0$}}
\newcommand{\bLo}{\mbox{\boldmath $L_0$}}
\newcommand{\bNo}{\mbox{\boldmath $N_0$}}
\newcommand{\bNr}{\mbox{\boldmath $N_R$}}
\newcommand{\bSi}{\mbox{\boldmath $S_i$}}
\newcommand{\bSd}{\mbox{\boldmath $S_d$}}
\newcommand{\bdSi}{\mbox{\boldmath $dS_i$}}
\newcommand{\bdSd}{\mbox{\boldmath $dS_d$}}
\newcommand{\bS}{\mbox{\boldmath $S$}}
\newcommand{\bdS}{\mbox{\boldmath $dS$}}
\newcommand{\bdt}{\mbox{\boldmath $dt$}}
\newcommand{\bdSdt}{\mbox{$\displaystyle \frac{\bdS}{\bdt}$}}
\newcommand{\bdSddt}{\mbox{$\displaystyle \frac{\bdS_d}{\bdt}$}}
\newcommand{\bdSidt}{\mbox{$\displaystyle \frac{\bdS_i}{\bdt}$}}
\newcommand{\bv}{\mbox{\boldmath $v$}}
\newcommand{\bp}{\mbox{\boldmath $p$}}

\parindent=0pt
\parskip=4pt

\begin{document}


\setlength\fboxsep{18pt}
\begin{boxedminipage}[hbp]{14cm}
{\bfseries Box 1. Reaction Network} Consider the simple reaction
network shown on the left below:

\begin{minipage}[t]{6cm}
\begin{picture}(60,95)(-60,-50)\thicklines
%
\put(18,20){$v_2$}
\qbezier(0,0)(20,20),(40,0) \put(42,-1){\vector(4,-3){2}}
%
\put(16,-46){$v_3$}
\qbezier(0,-20)(20,-40),(40,-20) \put(0,-19){\vector(-3,4){2}}
%
\put(-9,-12){$ES$}
\put(40,-13){$E$}
\put(65,23){$S_1$}
\put(65,-48){$S_2$}
\put(69,16){\vector(0,-1){48}}
\put(72,-8){$v_1$}
%
% Lower arm
\qbezier(62,-45)(20,-40),(0,-20)
% Upper arm
\qbezier(18,10)(26,8),(60,25) \put(61,25){\vector(2,1){2}}
%
\end{picture}
\end{minipage}
\begin{minipage}[t]{6cm}
\begin{picture}(60,85)(-40,-35)\thicklines
  % Vertical arrows
  \put(-9, 24){$ES$}
  \put(-9, 8){$S_1$}
  \put(-9, -11){$S_2$}
  \put(-9, -26){$E$}

  \put(26,40){$v_1$}
  \put(46,40){$v_2$}
  \put(68,40){$v_3$}

$\begin{array}{cc}
        \left[
         \begin{array}{rrr}
            0 & -1 &  1  \\
           -1 &  1 &  0  \\
            1 &  0 & -1  \\
            0 &  1 & -1
         \end{array}
        \right]
   \end{array}$
\end{picture}
\end{minipage}

\vspace{12pt} The {\bfseries stoichiometry matrix} for this
network is shown to the right. This network possesses two
conserved cycles given by the constraints: $S_1 + S_2 + ES = T_1$
and $E + ES = T_2$. The set of independent species includes:
$\{ES, S_1\}$ and the set of dependent species $\{E, S_2 \}$.

\bigskip
The $\bLo$ matrix can be shown to be:

\hspace{2cm} $$ \bLo = \left[ \begin{array}{rr}
  -1 & -1 \\
  -1 & 0 \\
\end{array}  \right]$$ \hfill
%
\linebreak
The complete set of equations for this model is therefore:

$$
\left[ \begin{array}{l}
  S_2 \\
  E \\
\end{array} \right] =
\left[ \begin{array}{rr}
  -1 & -1 \\
  -1 & 0 \\
\end{array}  \right]
\left[
\begin{array}{l}
    ES \\
    S_1 \\
\end{array}
\right] +
\left[
\begin{array}{l}
  T_1 \\
  T_2 \\
\end{array}
\right]$$
$$\left[
\begin{array}{l}
  dES/dt \\
  dS_1/dt \\
\end{array}
\right] =
\left[
\begin{array}{rrr}
  0 & -1 & 1 \\
  -1 & 1 & 0 \\
\end{array}
\right]
\left[
\begin{array}{l}
  v_1 \\
  v_2 \\
  v_3 \\
\end{array}
\right]$$ Note that even though there appears to be four variables
in this system, there are in fact only two independent variables, $\{ES, S_1\}$,
and hence only two differential equations and two linear constraints.

\end{boxedminipage}

\end{document} 