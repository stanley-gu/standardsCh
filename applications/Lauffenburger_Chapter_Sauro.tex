\documentclass[12pt]{article}
\usepackage{url}
\usepackage{graphicx}
\usepackage{amsmath}
\usepackage{setspace}
\usepackage{boxedminipage}
\usepackage[round,authoryear]{natbib}
\onehalfspacing
\allowdisplaybreaks[1]

\parindent=0pt
\parskip=4pt

\textheight     9in \topmargin      0.0in
\voffset -0.25in

\title{Software Tools for Systems Biology}
\author{Herbert M.\ Sauro and Frank T.\ Bergmann \\ \\ Department of Bioengineering, \\ University of Washington, Seattle, 98195-5061, WA}

\newcommand{\bgamma}{\mbox{\boldmath $\Gamma$}}
\newcommand{\bL}{\mbox{\boldmath $L$}}
\newcommand{\bT}{\mbox{\boldmath $T$}}
\newcommand{\bI}{\mbox{\boldmath $I$}}
\newcommand{\bM}{\mbox{\boldmath $M$}}
\newcommand{\bN}{\mbox{\boldmath $N$}}
\newcommand{\bE}{\mbox{\boldmath $E$}}
\newcommand{\bA}{\mbox{\boldmath $A$}}
\newcommand{\bB}{\mbox{\boldmath $B$}}
\newcommand{\bs}{\mbox{\boldmath $s$}}
\newcommand{\bK}{\mbox{\boldmath $K$}}
\newcommand{\bP}{\mbox{\boldmath $P$}}
\newcommand{\bR}{\mbox{\boldmath $R$}}
\newcommand{\bJ}{\mbox{\boldmath $J$}}
\newcommand{\bp}{\mbox{\boldmath $p$}}
\newcommand{\bx}{\mbox{\boldmath $x$}}
\newcommand{\bU}{\mbox{\boldmath $U$}}
\newcommand{\bV}{\mbox{\boldmath $V$}}
\newcommand{\bH}{\mbox{\boldmath $H$}}
\newcommand{\bZero}{\mbox{\boldmath $0$}}
\newcommand{\bLo}{\mbox{\boldmath $L_0$}}
\newcommand{\bNo}{\mbox{\boldmath $N_0$}}
\newcommand{\bNr}{\mbox{\boldmath $N_R$}}
\newcommand{\bSi}{\mbox{\boldmath $S_i$}}
\newcommand{\bSd}{\mbox{\boldmath $S_d$}}
\newcommand{\bdSi}{\mbox{\boldmath $dS_i$}}
\newcommand{\bdSd}{\mbox{\boldmath $dS_d$}}
\newcommand{\bS}{\mbox{\boldmath $S$}}
\newcommand{\bdS}{\mbox{\boldmath $dS$}}
\newcommand{\bds}{\mbox{\boldmath $ds$}}

\newcommand{\bdx}{\mbox{\boldmath $dx$}}
\newcommand{\bxi}{\mbox{\boldmath $x_i$}}
\newcommand{\bxd}{\mbox{\boldmath $x_d$}}
\newcommand{\bdxi}{\mbox{\boldmath $dx_i$}}
\newcommand{\bdxd}{\mbox{\boldmath $dx_d$}}

\newcommand{\bu}{\mbox{\boldmath $u$}}
\newcommand{\bdt}{\mbox{\boldmath $dt$}}
\newcommand{\bdvds}{\frac{\partial \bv}{\partial \bs}}
\newcommand{\bdvdp}{\frac{\partial \bv}{\partial \bp}}
\newcommand{\bdSdt}{\mbox{$\displaystyle \frac{\bdS}{\bdt}$}}
\newcommand{\bdSidt}{\mbox{$\displaystyle \frac{\bdS_i}{\bdt}$}}
\newcommand{\bv}{\mbox{\boldmath $v$}}
\newcommand{\el}[2]{\varepsilon^{#1}_{#2}}
\newcommand{\uel}[2]{\widetilde{\varepsilon^{#1}_{#2}}}

\begin{document}
\maketitle

\pagebreak

\tableofcontents

\pagebreak

\section*{Abstract}

Probably one of the most characteristic features of a living system is its continual propensity to change as it juggles the demands of survival with the need to replicate. Internally these changes are manifest by changes in metabolite, protein and gene activities. Such changes have become increasingly obvious to experimentalists with the advent of quantitative high-throughput technologies. Given the complexity of cellular networks it is no surprise that researchers have also turned to computer simulation
and the development of more theory based approaches to augment and assist in the development of a fully quantitative understanding of cellular dynamics. In this chapter we highlight some of the software tools and emerging standards for representing, simulating and analyzing cellular networks. The chapter will not be concerned with tools for managing high throughput data or analyzing genome scale data using bioinformatic approaches.

Abbreviations:

SBML: Systems Biology Markup Language \\
CellML: Cellular Markup Language \\
SBW: Systems Biology Workbench \\
MCA: Metabolic Control Analysis \\
FBA: Flux Balance Analysis \\
VCell: Virtual Cell \\
API: Application Programming Interface \\
XML: Extensible Markup Language \\
SOSLib: SBML ODE Solver Library \\
ODE: Ordinary Differential Equation \\
PDE: Partial Differential Equation \\


\pagebreak

\section{Introduction}

The use of simulation and theoretical studies in cellular and
molecular biology has a long, if uneven, tradition spanning almost
seventy years~\citep{Wright1934}. With the recent advent of
high-throughput technologies, interest in using more quantitative
approaches to enhance our understanding of cellular dynamics has
increased dramatically~\citep{TysonNatReview2001,neves:2002,GevaZatorsky:2006,Kholodenko:2006}.
As part of this trend there has been a significant rise in the availability of computer
software to help us model, simulate and analyze dynamic models of
cellular processes. In this chapter we will briefly survey the
available simulation software for systems biology. In addition we
will also briefly discuss model databases, model exchange standards,
ontologies and computational approaches that are relevant to the
study of dynamics in biochemical networks. We will not be concerned with software tools for managing high throughput data or analyzing genome scale data using bioinformatic approaches.

%Cellular networks are by nature highly dynamic structures where proteins, metabolites and expression levels constantly %change and adjust to internal and external influences. Due to the large number of components and nonlinear nature of the %interactions, the experimental study of network dynamics is often augmented with simulation and mathematical analysis .

Simulating biochemical networks has a long history dating back to at
least the 1940s~\citep{chance:1943}. The earliest simulations relied on
building either mechanical or electrical analogs of biochemical
networks. It was only in the late 1950s, with the advent of digital
computers and the development of specialized software tools~\citep{Ga68} that the ability to simulate biochemical networks became more widely available. In the intervening years up to 1980, a
handful of other software applications were developed~\citep{Burns1969,Bu71,PW73}
to help the small community of modelers. In more recent years, particularly
since the early 1990s, there has been a significant increase in interest in
modeling biochemical processes and a wide range of tools is now
available to the budding systems biologist. Many tools have been
developed by practicing scientists and are therefore available for
free (often in the form of open source) while others are commercial
such as SimBiology from MathWorks (\url{http://www.mathworks.com/}, 2006).

Given the recent explosion in interest in quantitative
biology~\citep{Klipp:2005,AlonBook}, the number and variety of tools and
approaches has likewise risen. In order to make the coverage more
manageable we will restrict our discussion to software that is aimed
at modeling systems based on differential equations or stochastic
chemical kinetics. We will not therefore be concerned software for modeling Boolean models, agent based models or models that use process calculi. For a discussion of
these models and related software the reader is referred to the
recent review by Fisher and Henzinger~\citep{Fisher2007} and~\cite{Degenring:2004}.

Moreover we will largely confine ourselves to dynamic models which means that we will not discuss at any length software that is designed to study the structural properties of networks other than in a few notable cases. In particular software used to build and compute flux balance models~\citep{FSm86a,Kauffman:2003,Papin:2006,PalssonBook:2007} or software for building isotopomer models~\citep{Wiechert:2001,Schwender:2008} will not be covered in detail even though such analyses are crucial to the metabolic engineering community.

Finally we will also not discuss tools that specialize in simulating spatial models or neurophysiology systems. For the former the interested reader should consult the following references for more information on this topic~\citep{ECELL,Ander:2004,Broderick:2005,Hattne:2005,Coggan:2005,Lemerle:2005,Sanford:2006}.
For simulating neural systems the reader can consult~\citep{Neuron:1993,NeuronBook:2006,Genesis:1998,kinetikit:2002}.
Although this might seem like a long list of exclusions, we are
still left with over one hundred tools that focus on simulating
non-spatial, deterministic or stochastic based models.

\subsection{The Problem}

The software discussed in this chapter is concerned with
analyzing two particular types of problem. One is the classical
problem of solving systems of differential equations that describe the deterministic
evolution of molecular species concentrations in time. These systems
are generally written in the form:
%
\begin{equation}
\frac{\bdx}{\bdt} = \bN\ \bv (\bx (\bp), \bp) \label{eq:system}
\end{equation}
%
where $\bx$ is the vector of species concentrations, $\bN$ the
stoichiometry matrix, $\bv$ the rate vector and $\bp$ the vector of
parameters. Often, the system of equations is overdetermined due to
the presence of conserved moieties~\cite{Re81} and the equation is then
reexpressed in the form:
%
{ \addtolength{\jot}{6pt}
\begin{eqnarray}
  \bxd &=& \bLo \bxi + \bT \nonumber \\
  \frac{d\bx}{dt} &=& \bNr\ \bv (\bx (\bp), \bp) \label{eq:general}
\end{eqnarray} }
%
where $\bxi$ is the vector of independent species, $\bxd$ is the
vector of dependent species, $\bT$ is the vector of conserved mass
totals, $\bNr$ is the reduced stoichiometry matrix and $\bLo$ is
part of the Link matrix~\citep{Re88a,SauroIngalls:2004}. If the
system (\ref{eq:system}) is overdetermined then it is highly
advisable to eliminate the redundant equations. There are a number
of reasons for this, the most obvious being the reduction in the
size of the model that accompanies the elimination. A more important
reason is that the Jacobian matrix, a fundamental quantity in
dynamical theory, is singular~\citep{Ravi:2006} in an over-determined
system. This in turn makes many important numerical methods
unstable. Many simulators fail to carry out this important stage.

The second problem of interest is the realization of the master
equation~\citep{Wilkinson:Book}. This equation describes the time
evolution of the concentration probabilities. However, since we need
one time evolution equation for every particle state in the system, the
system of equations becomes unwieldy very rapidly as the system size
increases. Instead modelers employ the exact stochastic simulation
(SSA,~\citet{Gill:1976}) algorithm (and it variants) developed by
Gillespie. Equation~(\ref{eq:stochastic}) summarizes this approach
where $j$ indicates which reaction will fire and at what time in the
future ($\tau$).
%
\begin{equation}
p (\tau, j\ |\ \bx, t) = a_j(\bx)\ e^{-a_o(\bx)\ \tau} \label{eq:stochastic}
\end{equation}
%
$t$ is the current time, $\tau$ is the time to the next
reaction, $\bx$ is the species vector, $a_j$ is the propensity function of reaction $j$ and $a_o$
the sum of all propensity functions. On its own, the SSA only
generates a single trajectory or realization. In order to glean
statistics on the model dynamics many repeated runs must be made
using this equation. As with the deterministic case, the system is
overdetermined and it is possible to reduce the work load by solving
explicitly for the independent species only.

A third approach to modeling, which is intermediate between~(\ref{eq:system}) and~(\ref{eq:stochastic}), is to employ the Langevin equation, that is a stochastic differential equation~\citep{StochasticDiff:2001}. A stochastic differential equation can be constructed by appending a noise term to the deterministic formulation (\ref{eq:system}), however very few tools currently provide facilities to solve
stochastic differential equation which is unfortunate~\citep{Adalsteinsson:2004,ChickarmaneP53:2007}.

Further information on modeling intracellular noise can be found in
the excellent review by~\citet{RaoNature2002}.

\subsection{The Need for Software}

The first question one might ask is why develop specialized software
to model biochemical networks? Given the availability of both
generic commercial and freely available tools for numerical analysis
one might ask if there is such a need. There are probably at least
two reasons why researchers develop their own specialized tools for
modeling biochemical systems. The first reason is that specialized
tools reduce the errors that occur when transcribing a reaction
scheme (that is a biological representation) into the mathematical
formalism ready for simulation. Deriving the math equation by hand is often a frequent source of error (especially in published papers) particularly in large models. The second important reason is that developing software offers an opportunity to codify and develop new numerical algorithms or new theoretical approaches that are specific
to problems found in systems biology. Such examples include
incorporating metabolic control analysis, MCA, ~\citep{KB73,SaCurrTop72}
into software (which is the reason why one of us developed
simulation software in the first place) or developing more efficient
methods for network analysis~\citep{Ravi:2006}. Although it is
possible to carry many analyzes in tools such as Matlab, Mathematica
or SciLab, the process tends to be more tedious and error prone and
certain analyses are unavailable to the user without applying considerable
effort.

\subsection{Functionality}

Given the long history of provision of software in modeling
biochemical networks, we believe there is a minimum level of
functionality that any newly developed or existing tool should have.
Foremost is the ability to convert a biological description, that is
a set of biochemical reactions and corresponding rate laws, into a
set of differential equations. The input format for the model can be
in the form of SBML (Systems Biology Markup
Language, \citep{hucka:2002d} or CellML~\citep{hedley:2001b} - (Cell Markup Language), see
section 4 for details - or it can be in the form of a human readable
text format which can be more easily edited and, if necessary,
converted later into SBML or CellML for export. Tools such as the
Systems Biology Workbench (SBW, see section 2.7 for details) provide
facilities for converting SBML to simple human readable text files
which can be edited and loaded into the simulation tool.

Once a model has been converted into a set of appropriate
differential equations, the solutions can be easily generated from
widely available numerical integration routines. The US Department of Energy in
particular has developed a very useful series of sophisticated
libraries, including SUNDIALS~\citep{CVODE:2005} and
ODEPACK~\citep{HIND83}. These libraries are quite straight forward to
incorporate into software and can be used to solve even some of the
most complex problems.

Together with generating a solution, some means to save simulation
results is obviously needed and in a format that can be loaded by
widely available tools such as Gnuplot~\citep{williams:1998} or Excel (\url{microsoft.com}). The final minimum
requirement is some way to modify parameters and initial conditions
to allow a user to repeat runs although such functionality could
simply involve editing the model in a text file and rerunning the
model.

What was just described is essentially the functionality of some of
the earliest simulators dating back to the 1950s. Clearly, any
simulator developed today should exceed these requirements.

Beyond the minimum requirements there are a whole host of other
desirable capabilities, these include: model reduction by way of
conservation laws, steady state determination, sensitivity analysis
(via MCA is possible), frequency analysis, bifurcation discovery and
analysis, structural analysis, and finally optimization and
parameter fitting. In addition to these, other capabilities such as
access to a model database (particularly the BioModels Database~\cite{BioModels:2006}. see section 5),
the availability of a variety of model editors (visual, text and
GUI) and finally some way to generate camera ready copy of network
diagrams for publication purposes would also be desirable.

\section{A Brief Survey of Commonly Used Tools}

In the last ten years or so, numerous software applications have
been written to support modeling of biochemical processes. The
variety of software is huge even within the restricted scope of this
discussion. Almost every computer language has been used at one time
or another to build simulation software. Some applications work on
single operating platforms, others work across multiple platforms;
some are trivial to install, while others require considerable
persuasive efforts to make them work. There is also a wide variety
of user interfaces such as graphical user interfaces incorporating
menus, buttons, lists etc., network design interfaces that allow
users to draw a pathway on the screen and text based scripts that
enable users to control and define a computer model very precisely.
Finally, application capabilities vary enormously, some attempt to
be broad offering many different capabilities, others choose to
specialize in one particular area. Some of the tools are
extensible, that is new functionality can be added by a user either
through scripting languages or by writing small plugin applications.
Most have some form of reference documentation, including tutorials
to help users get started.

A visit to the software guide at the SBML.org web site will reveal
over one hundred different tools for modeling or working with biochemical networks.
The majority of these have been developed in the last eight years or
so. However, although there appears to be a large number of
offerings, many are not actively developed and have become orphaned
software. In this chapter we will only focus on those tools which
are actively developed and have an excellent chance of being
supported in the future.

We will focus initially one some tools we use ourselves. We
sometimes use these tools because our software may lack certain
features we need. More often we also use these tools to compare
simulation results.

Before continuing we would also like to mention
three legacy simulators which have influenced the development of
most modern applications, these include Gepasi~\citep{Gepasi:1993},
METAMOD~\citep{HM86} and SCAMP~\citep{SauroF91,SauroScamp:1993}. All
three were developed in the 1980s and in some cases continue to be
used.

All the tools we will describe can import SBML, and all, with the exception of JSim, can export
SBML, however JSim can also import CellML.
Both SBML and CellML are established model exchange standards which
we will discuss in more detail in a later section. All the tools we
will describe are currently maintained and have an active user
community. Most of the tools run on multiple platforms (Windows,
Linux, Mac) and most, with the exception of one or two, are open
source though under varying licenses.

Given that we are also developers of simulation tools it would be
unfair for us to provide a detailed comparison of the different
applications, instead we refer interested readers to recent
independent reviews by Alves~\citep{Alves:2006},
Manninen~\citep{Manninen:2006} and Vacheva~\citep{VachevaE:2006} and
two comparison web sites, one maintained by the VCell group at
\url{http://ntcnp.org/twiki/bin/view/VCell/SBMLFeatureToolMatrix}
and another by our own group at
\url{http://www.sys-bio.org/sbwWiki/compare}. The feature matrix at
the VCell site is a user contributed table that lists the
capabilities of each tool. In contrast, the comparison site at
sys-bio.org does not attempt to make statements about particular
capabilities instead it runs each registered tool through every
model in the BioModels Database (over 170 models) and compares the
results. From such results, a developer or user can judge the
capabilities of a particular software tool without any bias from
ourselves.

Although not on the list, we should mention the use of Matlab as an
excellent numerical and data analysis application in systems
biology. Matlab is an application which we ourselves have frequently
used for complex data analysis. MathWorks also supplies a specialized
tool box called SimBiology which offers many useful capabilities.

All these tools that we will now describe are capable of basic
simulation, that is solving differential equations or simulating a
stochastic model, what makes them stand out is what they add in
addition to this basic functionality.


%Other tools that are particularly important, though do require further development, are bifurcation analysis %software. Bifurcation analysis, the study of model behavior as a function of parameters is probably one of the %most important analysis methods available and we include in this list XPP and
%Oscill8 as good candidates. Other tools that we use include Copasi and PySCeS~\cite{Pysces2005} for testing %purposes, both these tools are developed by long stand-ing researchers in the field and we consider their tools to %be reliable. For most of our work we use SBW, with Jarnac and JDesigner being the most important tools in
%this category. Every research group will have their favorites and our own particular choices are biased according %to our own interests and needs.

%\medskip
%\begin{table}
%\label{table:Apps}
%\begin{tabular}{ll}
%\rowcolor[gray]{0.8}\color{black}
%Application & Description \\ \\
%Copasi & This is an excellent tool with a very wide range \\
%       & of capabilities. If uses a menu/dialog based \\
%       & user interface. It is open source and is \\
%       & cross-platform~\cite{Copasi2006}. \\
%       & \\
%Jarnac/JDesigner & Jarnac is a script based application, \\
%      & JDesigner is a visual design tool which can use \\
%      & Jarnac via SBW to carry out simulations. The simulation \\
%      & capabilities of Jarnac are quite extensive, offering \\
%      & both ODE and stochastic simulation - Open Source, \\
%      & Windows, Linux \cite{sauro:2000,Sauro:Omics}. \\
%      & \\
%JSim & JSim is Matlab like application that allows users to \\
%       & \\
%     & describe models mathematically. It's unique capability \\
%     & is unit checking to ensure dimensional consistency \\
%     & \\
%PySCeS & This is a very complete ODE based simulation \\
%       & environment built around the scripting language \\
%       & Python - Open Source, multiplatform \\
%       & \cite{Pysces2005}. \\
%       & \\
%       & \\
%VCell  & A very mature server based application that is \\
%       & specialized to build and simulate large scale \\
%       & spatial (PDE) models - Open Source, multiplatform \\
%       & \cite{VCELL}. \\
%\hline
%\end{tabular}
%\caption{Mature and Easily Accessible Tools for Modeling Cellular Networks.}\label{tbl:Tools}
%\end{table}

\subsection{COPASI}

Pedro Mendes wrote one of the earliest PC simulators which he called
Gepasi \citep{Gepasi:1993}. COPASI~\citep{Copasi2006} is essentially a rewrite of
Gepasi, that comes in two versions: a graphical user interface and a
command line version. The command line version is designed for batch
jobs where a graphical user interface is unnecessary and where runs
can be carried out without human supervision. COPASI uses its own
file format to store models, however like all the tools discussed
here, it can import and export SBML. One of its undoubted strengths
is optimization and parameter fitting which it inherited from its
predecessor. It has an unique ability to optimize on a great variety
of different criteria including metrics such as eigenvalues,
transient times etc. This makes COPASI extremely flexible for
optimization problems. Installation is very simple and entails using
a one-click installer. Although the source code to COPASI is
available and can be freely used for research purposes in academia,
owing to the way in which the development of COPASI was funded there
are restrictions on commercial use.

The graphical user interface is based on a menu/dialog approach,
much like its immediate predecessor, Gepasi. COPASI has
capabilities to simulate deterministic as well as stochastic models
and includes a wide range of analyzes. It correctly takes into
account conservation laws and has very good support for metabolic
control analysis amongst other things. COPASI is without doubt one
of the better simulators available. Although the user interface is
graphical, it does, due to its particular design, require some effort
to master but with the availability of the COPASI the source code,
there is the opportunity to provide alternative user interfaces.. Finally there is a
version that has an SBW interface (Systems Biology Workbench) which
allows SBW enabled tools access to COPASI's functionality (currently available at \url{sys-bio.org}).

\subsection{Jarnac}

Jarnac~\citep{sauro:2000} is a rapid prototyping script based tool
that was developed as a successor to SCAMP~\citep{SauroF91}. It is
distributed as part of the Systems Biology Workbench which makes
installation a on-click affair. Jarnac was developed in the late
1990s before the advent of portable GUI toolkits which explains why
it only runs under Windows although it runs well under Wine
(Windows emulator) thus permitting it to run under Linux. Visually,
Jarnac has two main windows, a console where commands can be issued
and results returned and an editor where control scripts and models
can be developed. The application also has a plotting window which
is used when graphing commands are issued.

Jarnac implements two languages, a biochemical
descriptive language which allows users to enter models as reaction
schemes (similar to a SCAMP script) and a second language, the
model control language which is a full featured scripting language
that can be used to manipulate and analyze a model. The main
advantage of Jarnac over other tools is that models can be very
rapidly built and modeled. From our experience with using many
simulation tools over the years, Jarnac probably offers the fastest
development time for model building of any tool. Models can also be
imported or exported as SBML. Like COPASI and PySCeS, Jarnac offers
many analysis capabilities including extensive support for metabolic
control analysis, structural analysis of networks and stochastic
simulation. It has no explicit support for parameter fitting but
this is easily remedied by transferring a model directly to a tool
such as COPASI via SBW.

\subsection{JSim}

JSim~\citep{JSim:2003} is a Java-based simulation system for building
quantitative numeric models and analyzing them with respect to
experimental reference data. JSim can either be used from a web
browser or downloaded to a desktop machine. JSim's primary focus
is physiology and biomedicine, however its computational abilities
are quite general and applicable to a wide range of scientific
domains. JSim models may intermix ODEs, PDEs, implicit equations,
integrals, summations, discrete events and procedural code as
appropriate. One of JSim's strengths which makes it standout from
other simulators is its careful attention to dimensional analysis.
Users can thus specify units for all terms in a model and JSim will
then carry out an automatic check to ensure that the specified units
are consistent with the model. In addition JSim can automatically
insert conversion factors for compatible physical units. JSim is one
of the few modeling programs that can import SBML as well as CellML
formats.

JSim's modeling language is mathematically orientated and thus may not be suited to all users, however the authors also define a more biologically orientated language called BCL which allows users to specify models in a more biological orientated manner.

\subsection{MathSBML}

MathSBML~\citep{Shapiro:2004} is an open-source package for working with SBML models in Mathematica. It provides facilities for reading SBML models, converting them to systems of ordinary differential equations for simulation and plotting in Mathematica, and translating the models to other formats. MathSBML is one of the few simulators to support differential/algebraic equations that may present themselves in models. For Mathematica users, MathSBML is a good choice.

\subsection{PySCeS}

PySCeS~\citep{Pysces2005} is probably the best Python based simulator
currently available. Although users interact with PySCeS via Python
scripts, many of the underlying numerical capabilities are provided
by well established C/C++ or FORTRAN based numerical libraries. It
was written by a team well seasoned in biochemical modeling and as a
result the software is reliable and comprehensive. As a portable
free scripting tool for systems biology, PySCeS is currently one of
the best. The only major element missing is parameter optimization
but such functionality can be added through additional Python
scripting. As with COPASI and Jarnac, PySCeS correctly handles
conservation laws and has very extensive support for metabolic
control analysis including extensive structural analysis
capabilities. PySCeS is also one of the few tools to have some
limited support for bifurcation analysis. Installation on Windows is
very straight forward with a single click install. Although the
current version of PySCeS is mainly concerned with deterministic
simulations, there are plans to add stochastic capabilities in
collaboration with the University of Newcastle, UK. There are a
number of other Python based simulators including
sloppyCell~\citep{SloppyCell:2007}, Scrumpy~\citep{Poolman:2006}, ByoDyn~(\url{http://cbbl.imim.es:8080/ByoDyn}) and PyLESS (\url{http://sysbio.molgen.mpg.de/pyless/}) which specializes
in model reduction. PySCeS however is the most mature and comprehensive of these.

\subsection{SBToolbox$^2$}

This is a very extensive Matlab tool box developed by Henning
Schmidt \citep{Henning:2006}. The tool box has a wide range of
capabilities. One of it unique features is the ability to interact
with XPPAUT~\citep{XPPAUT:2002} by generating the XPP files and
launching XPPAUT. For a Matlab user the SBToolbox$^2$ is a
particularly useful tool to have available.

\subsection{Systems Biology Workbench}

The Systems Biology Workbench~\citep{Sauro:Omics,BergmannCP:2006},
unlike other tools described in this chapter is not a simulation
resource itself, it is instead an open source framework for
connecting heterogeneous software applications. SBW is made up of
two kinds of components:

\begin{itemize}
\item Modules: These are the applications that a user would use. There is available a broad collection of
model editing, model simulation and model analysis tools.
\item Framework: The software framework that allows developers to cross programming language boundaries and
connect application modules to form new applications.
\end{itemize}

Modules in SBW can be written in a wide variety of languages
including C/C++, Java, .NET, Delphi, Matlab, Python and Perl. Each
module will expose application programming interfaces that allow
other application modules access to their functionality.

One of the core modules in SBW is the simulation module, roadRunner.
roadRunner is implemented using the C\# programming language and
exploits the ability of .NET to generate model code on the fly and
subsequent compilation and optimization by the .NET framework. This
approach leads to the generation of very fast simulation times.
Numerical analysis is provided by the CVODE~\citep{CVODE:2005}
integrator and the NLEQ
library~(\url{http://www.zib.de/Numerik/numsoft/NewtonLib/}) for
steady state computations. RoadRunner is available on the Windows
platforms through the .NET framework and on POSIX systems through
the mono project~(\url{http://www.mono-project.com}). roadRunner
relies on several SBW modules for the reading of SBML and for
conservation analysis, resulting in a smaller set of ODEs to solve
but also the ability to generate non-singular Jacobian matrices
which is required for steady state evaluation and other analyses. In
addition to simulation and steady state evaluation, roadRunner also
incorporates a full set of routines to compute sensitivities (i.e.\
Metabolic Control Analysis), frequency analysis, various structural
metrics from the stoichiometry matrix and a basic continuation
algorithm based on arc length~\citep{Kub76,Kub83}. RoadRunner is also the backend simulator for the
online simulation tools at \url{sys-bio.org}.

\subsection{VCell}

The Virtual Cell~\citep[VCell,][]{Schaff:1997,VCell:2002,VCell2003} is
a client/server based tool that specializes in three-dimensional
cell simulations. It is unique in that it provides a framework for
not only modeling biochemical networks but also
electrophysiological, and transport phenomena while at the same time
considering the subcellular localization of the molecules that take
part in them. This localization can take the form of a
three-dimensional arbitrarily shaped cell, where the molecular
species might be heterogeneously distributed. In addition, the
geometry of the cell, including the locations and shapes of
subcellular organelles, can be imported directly from microscope
images. VCell is written in Java but has numerical analysis carried
out by C/C++ and FORTRAN coded software to improve performance.
Currently, modeling must be carried out using the client/server
model which necessitates a connection to the internet. In addition
models are generally stored on the VCell remote server rather than
the clients desktop. This operating model is not always agreeable to
users and as a result the VCell team are reorganizing the software
so that it can also be run as a stand-alone application on a
researchers machine. Recently the VCell team has incorporated the
BioNetGen~\citep{Blinov:2004} network generator which allows models
to be specified in a rule based manner. VCell is also one of the few
tools that can both import and export SBML and CellML. This feature
could in principle be used to translate between SBML and CellML models.

\subsection{Visual Editors}

Tools that allow users to draw pathways on a screen and turn them into simulatable models seem to be fairly rare. We confine our attention here to tools that are specifically designed to assist in simulation rather than pathway annotation. Examples of the later include the Edinburgh Pathway Editor~\citep{EdinPathway:2006}, cytoscape~\citep{Shannon:2003}, BioUML~\citep{kolpakov:2004} and BioTapestry~\citep{longabaugh:2005} but many others exist. We mention the later tools specifically because they have some simulation capability but their strengths lie elsewhere. A very complete review of non-simulation based network viewers can be found in~\citet{Suderman:2007}.

Probably one of the first visual editors to be written for
simulation was a Mac based tool called
KineCyte~\citep{Cook:1997,Cook:2001} by Dan Cook and around the same
time JDesigner~\citep{BergmannCP:2006} was developed by Sauro for the
Windows platform. Other tools include applications such as
SimWiz~\citep{Rost:2004}, CADLive~\citep{Kurata:2007} and
Cellware~\citep{Dhar:2004}, although unfortunately Cellware does not
appear to be under development any longer. The three main visual
design tools still supported and geared towards simulation are
JDesigner, ProMoT~\citep{Ginkel:2003} and CellDesigner~\citep{Kitano:2005:Nat-Biotechnol}.

\subsubsection{JDesigner}

JDesigner~\citep{Sauro:Omics,BergmannCP:2006} is open source (BSD
licence) and runs under Windows. It requires SBW to enable
simulation capabilities. With SBW, models can be constructed using
JDesigner and seamlessly transferred to other tools such as COPASI
or any other SBW enabled tools. Unlike CellDesigner, JDesigner takes
a minimal approach to representing networks. CellDesigner has twelve
node types (plus variants) and six different transition types.
JDesigner in contrast has one node type, one generic reaction type
and two regulatory types. All networks can be constructed from these
four basic types. This minimal approach reflects the fact that the
underlying models are the same regardless of the molecules or
reaction types. Thus protein network models and metabolic models are
indistinguishable at the mathematical level. Although JDesigner has
only a limited number of types, nodes, reactions and membranes can
be modified visually to change colors, shapes etc. Moreover, nodes
can be decorated with covalent sites and multimeric structures.
JDesigner uses fully adjustable multi-bezier arcs to generate
reactions and regulatory arcs and has a variety of export formats
that allow camera-ready copy to be generated for publications.
Models are stored in native SBML with specific open access annotations to store
the visual information.

\subsubsection{CellDesigner}

CellDesigner~\citep{Kitano:2005:Nat-Biotechnol}, developed by Akira
Funahashi in the Kitano group, is a popular close-source but
cross-platform Java based tool. Like JDesigner it also needs
external support to enable simulation, which can include SBW or
SOSLib (\citep{Machne:2006}, see section 2.13.2). One of the main
strengths of CellDesigner is its visually appealing graphical
representation which has made CellDesigner a popular tool. Its
other strength, though still under development, is the availability
of a plug-in architecture which in principle will allow other
third-party programmers to develop add-ons to the
tool~\citep{andreas:2008}. One potential drawback of CellDesigner is
that the format that the tool uses to store the models is
undocumented so that only CellDesigner is capable of reading and
writing the models. Only one other tool, the visual layout tool in
SBW can read CellDesigner models and that was accomplished by
reverse engineering the CellDesigner format. With the advent of the
Systems Biology Graphical Notation it is hoped that tool writers
will move towards a community agreed and open format for
representing biochemical networks (See later section for more
details).

\subsubsection{Commercial Visual Design Tools}

There are also some commercial pathway editors including Cell
Illustrator~(\url{http://www.cellillustrator.com}), SimBiology from
MathWorks and an interesting contribution from Gene Network
Sciences~(\url{http://www.gnsbiotech.com/}) which utilized a visual
language called DCL (Diagrammatic cell language) which unfortunately
was patented and thereby sealed its future and in any case seems no
longer to be available.

\subsection{CellML based Applications}

Although SBML is by and large one of the more popular exchange
formats, CellML is also an important standard particularly among
cell physiologists, where models combine biochemical network models
with transport and electrophysiological behavior. For this reason
most applications developed with CellML import and export
capabilities tend to be geared towards physiological modeling and
often do no to support SBML. Hybrid applications such as VCell and
JSim try to support both formats with varying degrees of success
since the formats may not always be compatible.

We have already mentioned JSim as an application that can import
CellML models. In addition to JSim there are a number of prominent
simulators in this category. The most important of these include,
Andre's CellML Tools, COR~\citep{Garny:2003}, PCEnv and
VCell~\citep{VCell2003}. All these tools can import {\em and} export
CellML.

Andre's CellML tools include a series of utilities to browse CellML
models and carry out basic simulations. COR, another important
CellML based tool, is a Windows application developed by Alan Garny
at Oxford University in the UK and was probably the first simulator
to support CellML. COR uses an interesting scripting language to
describe a model which means users do not need to know the
intricacies of CellML. The language incorporates facilities to allow
a user to specify units, much like the facilities found in JSim. COR
also supports high speed simulation capabilities through runtime
compilation. Installation is very simple and uses a one-click
installer.

PCEnv is a tool that is being development by the CellML team at
Auckland. It incorporates a basic simulator permitting users to
generate time course simulations, change parameters and generate new
models. Model generation is GUI based where a user builds a tree
representing the various components of the model. In addition to the
GUI interface there is also a JavaScript interface which offers some
control over the output and simulation. Installation is very simple
and uses a one-click installer. Tools such as COR and PCEnv are
under very active development and it is hoped that in the future
they will offer many of the facilities that are currently available
in other systems biology simulation tools.

\subsection{Other Tools of Interest}

We list here some other tools that we have not mentioned but are
notable for one reason or another. Not all of these tools appear to
be actively developed but may experience continued developed in the
future. Most if not all of these tools support SBML import and
export. All these tools are worth looking at, some implement
particular functions very well. In the descriptions only a brief
hint is given concerning their capabilities and the reader is
strongly urged to investigate them further if some particular
functionality is of interest.

\begin{description}
%
\item[BioNetGen:] BioNetGen~\citep{Blinov:2004} is a tool for automatically generating
mathematical models of biological systems from user-specified rules
for biomolecular interactions, particularly signaling pathways.
Rules are specified in a script based language from which the
reaction scheme is generated and subsequently the mathematical
model. There are not many tools that operate on a rule based approach ({\em cf.} little b,
\url{http://www.littleb.org/}) but all of them take a particular and fairly radical view of
biology that assumes that all interactions are potentially
significant of which there may be many thousands in a given model
depending on the specified rules.
%
\item[Cellerator/xCellerator:] This is part of the SIGMOID project~\citep{Sigmoid:2005}
at Caltech and Irvine although its origins are earlier. It is a
Mathematica based tool designed to facilitate biological modeling
via automated equation generation. Two notable features of this tool
include the ability to embed arrow-based chemical notation directly
into Mathematica scripts and support for multicellular models. Model
analysis is fairly basic, confined largely to generating time course
simulations. The other Mathematica tool,
MathSBML~\citep{Shapiro:2004} has much more extensive analysis
capabilities.
%
\item[E-Cell:] The E-Cell~\citep{ECELL} simulation environment is an
object-oriented software suite for modeling, simulation, and
analysis of large scale complex systems such as biological cells. It
has been in development since 1996 and can, in the latest version,
be accessed via Python. For some reason E-Cell has never gained the popularity that perhaps is should have.
Early versions suffered from a difficult to use user interface which
most likely slowed its acceptance by the community and model interchange support has been weak which also hampered its acceptance by the wider community.
%
\item[JigCell:] JigCell~\citep{VassJigCell:2004} is a modeling tool
developed by the Tyson group who are well known for their cell cycle
models~\citep{TysonBioessay2002}. Many tools are deigned by
programmers rather than modelers but JigCell is one of those few
tools that was written specifically by a well established modeling
group to fit models to data and test the models against numerous
experimental results. The system works in a similar way to
regression testing in software design where the computational model
is repeatedly tested as the model evolves or new
experimental data is made available. This enables the generation of
robust and more reliable models. As part of this goal, JigCell also
supports parameter estimation and the ability to maintain an
ensemble of models. The unique functionality provided by JigCell reflects the
fact that the authors are modelers rather than just software developers.
%
\item[ProMoT:] ProMoT~\citep{Ginkel:2003} stands for process modeling tool and
originated from the process engineering community but has since
evolved to encompass biochemical modeling. Currently it only runs
well on Linux but plans have been made to make a Windows version. It
is notable for its ability to model differential-algebraic equations
and is one of the few tools to provide visual support for dealing
with model libraries, modules and hierarchical models.
%
\item[Oscill8:] Oscill8~(\url{http://oscill8.sf.net/}) is a
bifurcation analysis tool developed by the Emery Conrad at the Tyson
group. It runs well on the Windows platform and takes XPP scripts as
input. Oscill8 represents one of the few bifurcation packages with a
modern GUI interface. SBW supplies a special SBML to XPP translator
that is designed to work with Oscill8. Although the XPPAUT site also
provides an SBML to XPP translator this is unsuitable for many
biochemical models because it doesn't take into consideration
conservation laws in the models which renders the Jacobian singular
and thereby unsuitable for the AUTO numerical algorithms.

\item[PottersWheel:] This is a very comprehensive parameter
fitting tool~\citep{PottersWheel:2008} that works well with the
SBToolBox$^2$~\citep{Henning:2006} but can also be used alone. In a
number of cases it is better than COPASI's capabilities particularly
in the area of generating nonlinear confidence limits on parameter
fits and analyzing the resulting fit. The experimental data input
formats are also very flexible. The tool provides out of the box a
number of optimization algorithms including Genetic and simulated
annealing approaches. For a Matlab user interested in doing
parameter fitting properly, this tool is a serious contender
especially since the built-in optimization methods in Matlab itself
are not sufficiently robust for fitting biochemical models. A
further advantage of this tool is that is comes with copious
documentation. PottersWheel combined with the SBToolbox$^2$ is a
powerful mix.
%
\item[R Packages:] There is currently limited support for dynamic modeling in R, but this is changing. Given
the growing interest in using R in biological research it would not
be surprising if further work is done in this area. Currently the
two tools that are of interest include SBMLR package by Tomas
Radivoyevitch and a more modern version by Michael Lawrence called
R-SBML. Modeling is however in all cases limited to simple time
course simulations. Both packages use the ODEPACK library for
integrating the ODEs which arguably is superior to CVODE from
SUNDIALS due to ODEPACK's ability to switch integration modes during
the integration.
%
\item[SBML-PET:] SBML-PET~\citep{Zi:2006} is a Systems Biology Markup
Language (SBML) based Parameter Estimation Tool. Of interest is that
the tool supports not only event handling in SBML but also uses the
stochastic ranking evolution strategy (SRES,~\citet{SRES:2007}) for
fitting, which is arguably one of the best fitting strategies to
use~\citep{Moles:2003}.
%
\item[SBMLToolbox:] This is a Matlab tool box developed by the Caltech SBML
team~\citep{Keating:2006}. Its primary function is to load and save
SBML in to Matlab data structures. It uses the built-in Matlab
functions to carry out simulations. Its importance lies in it's
careful adherence to the SBML specifications. Other Matlab related tools add functionality to this toolbox.
%
\item[SloppyCell:] This is a Python based tool developed at Cornell~\citep{SloppyCell:2007},
it has some significant capabilities in parameter fitting related to
estimating confidence limits in fitted models. For a Python based
tool that supports parameter fitting this is probably the most
mature although the supplied optimization methods are limited in
scope (Levenberg-Marquardt~\citep{levenberg:1944,marquardt:1963} and Nelder and Mead~\citep{nelder:1965}). It does support ensemble model fitting which is very useful and is probably it's
main Raison d'etre. It does not however compare with PySCeS in other
areas. If additional fitting methods, for example, a stochastic
ranking evolution strategy (SRES,~\citet{SRES:2007}), were included, a
combination of sloppyCell and PySCeS would be extremely powerful.
%
\item[XPPAUT:] Although XPP~\citep{XPPAUT:2002} is not a tool that
was written specifically for the systems biology community,  its

great utility is its extensive support for bifurcation analysis, a
largely neglected field in the software development community
particulary in systems biology. XPP gains its computational
capabilities from a well established software library called
AUTO~\citep{Do81}. Although it is possible to use AUTO on its own,
XPP adds a user interface to make it easier to use. Although very
capable, XPP's user interface is however not very modern by current
standards and does take a little getting used to. Running XPP on
Windows is awkward and XPP is really meant to be run on the
Linux/Unix platform. For an easier tool, but perhaps with more
limited capability, Oscill8 is a good choice for the Windows user.
For Python users there is an interesting tool called
PyDSTool~\citep{SloppyCell:2007} (inspired by the original
DSTool~\citep{DSTool:1992}) although users must derive the
differential equations themselves and installation is by no means
trivial. However it may be possible to modify the tool for the
systems biology community.
\end{description}

\subsection{Commercial Tools}

There is a small but growing list of commercial titles available in the market. The most prominent of these include Berkeley Madonna, Jacobian, SimBiology from MathWorks and Terranode. We should also mention SimPheny from Genomatica even though it is not a dynamic simulation package but because it is a well established tool for flux balance analysis. For a non-commercial flux balance tool see~\citep{Klamt:2007} - note that this tool requires a commercial copy of Matlab to operate. All these tools can export SBML with the exception of Berkeley Madonna although COPASI can export Berkeley Madonna files. The one real issue with commercial tools is the lack of transparency with regard to the numerical algorithms they employ and the impossibility of changing the methods when necessary. MathWorks to its credit has documentation on the methods they use but the code itself is not visible to the researcher. This is a significant problem for a research based community, less so perhaps for a cooperate funded project. It is of course a difficult issue to reconcile for commercial vendors since the underlying numerical methods are often the source of value in the product.

\subsection{Dedicated Libraries}

Up to now we have focused on describing enduser applications where a user would download an installer and in a very short time be ready to start defining a model and running a simulation. As mentioned previously there are a huge number of such tools available, though sadly many are started but left unfinished. In addition many tools reinvent basic functionality over and over again wasting considerable human intellect and time. In this section we would like to describe a small but growing list of software libraries dedicated to dynamic simulation in systems biology and targeted at developers of new software. These libraries are cross-platform and most significantly cross-language solutions to common computational and data management problems in systems biology. They enable developers to create new applications from existing well tested and documented solutions and enable developers to focus on novel functionality rather than spending most of their effort on reinventing well understood themes. Furthermore, from a historical perspective, good, well tested software libraries tend to survive the turmoils of funding cycles and researcher turnover whereas enduser applications tend not to.


\subsubsection{libSBML}

LibSBML~\citep{libSBML:2008} is a very useful software library
provided by the SBML team that can be used to read and write SBML.
The success of SBML over competing standards can be ascribed in part
to libSBML. The software library is based around a C/C++ core, with
wrappers provided for many programming languages. Furthermore the
library is available for Windows and POSIX operating systems where a
C based API is exposed and thus can be linked to virtually any other
computer language with relative ease. With an abundance of
documentation and examples, software developers can readily use
libSBML for their SBML support.

By using libSBML a developer can focus on how to interpret
computational models rather than concerning themselves with the
mechanics of reading and writing SBML. At of the time of writing,
libSBML has released version 3. This version has new features
including the ability to validate the model, such as unit
consistency checking, or checks on whether the model assignment
expressions are over determined. LibSBML also provides support for
MIRIAM compatible annotations~\citep{miriam}. LibSBML is very well
developed library that sets a high standard for the development of
other software libraries in the systems biology community.


\subsubsection{SOSLib}

The SBML ODE Solver Library~\citep{Machne:2006} or SOSlib, is a
simulation library that can be linked against any application using
the C programming language. At the core, SOSlib uses the SUNDIALS
suite~\citep{CVODE:2005} and libSBML. Like most SBML capable
simulators it does not deal with models using delayed differential
equations and algebraic equations. On the other hand SOSlib is one
of the few simulators that can handle events reasonably well.
Supporting SBML events has proved to be a difficult undertaking for
many developers because of the numerical difficulties that they
generate. Very few ODE integrators support event detection and as a
result only very few simulators (such as SOSLib and roadRunner) are
capable of correctly handling them.

SOSlib can be run in two modes, in the interpreted mode, SOSlib will
use the abstract syntax trees of libSBML to evaluate the kinetic
laws and other equations. The use of the abstract syntax tree also
permits SOSLib to invoke symbolic evaluation when computing the
elements of the Jacobian matrix (a techniques also employed by
SCAMP~\citep{SauroF91,SauroScamp:1993}). The second mode is where the
right hand side of every ODE is compiled, using either GCC on POSIX
environments or tcc (\url{http://fabrice.bellard.free.fr/tcc/}) on
Windows. This allows SOSLib to generate fast simulation times.
SOSlib however has one significant drawback which is its inability
to remove dependent variables from the model. If this deficiency
could be corrected, SOSLib would be a much more useful library to
the community.

\subsubsection{libStructural}

The libStructural library is a software library for conservation and
structural analysis of stoichiometric networks and is written in
C/C++ for cross platform portability (\url{http://sys-bio.org/libStructural}).

One of the initial steps in a simulation (deterministic or
stochastic) is to consider the stoichiometric structure of a model.
This is carried out for a number of reasons, first it allows model
reduction to take place and permits faster simulation times,
secondly it allows users to study the structural characteristics of
the model, such as flux and moiety conservation. The flux
constraints can be further used to establish linear programming
models for flux balance analysis~\citep{FSm86a,PalssonBook:2007} and also as an aid to constructing the isotopomers dynamic models used in isotopic measurements of fluxes~\citep{Wiechert:2001,Schwender:2008}. The library provides over sixty API calls that gives information on various stoichiometric properties such as the kernel, link,
dependent and independent flux and species matrices. In addition the
library exposes some useful functionality from the LAPACK
library~\citep{Lapack:1999}. The library can also be built
against libSBML so that models in standard SBML can be analyzed
otherwise the library accepts raw stoichiometry matrices. The
library is fully documented and tested against the very large models
from the Palsson group~\citep{Price03} and over twenty smaller test models.


\section{Dedicated Tools for Stochastic Simulation}

Stochastic simulation has received much attention in the last ten to
twenty years owing to the realization that continuous models are not
always an appropriate description of what is, at the molecular
level, a fundamentally discrete process. The development of
practical algorithms for stochastic simulated owes its origin to a
series of classical papers published by Gillespie in the 1970s.
Gillespie's approach yielded an algorithm that was extremely
straight forward to implement in software but was largely unknown
until the early 1990s in biology. Two of the first papers to utilize
Gillespie's approach in biology are Kraus~\citep{Kraus:1992} and
Moniz-Barreto~\citep{MonizBarreto:1993}. However it was probably the
paper by Arkin et al.~\citep{ArkinLambda1998} that convinced many of
the importance of stochastic dynamics in biology. In the intervening
years there has been an explosion in interest in stochastic
dynamics, attracting researchers from many diverse
fields~\citep{PiCalc:2003,Wilkinson:Book,Ullah:2007}.

Together with modeling efforts there have been developments in software particulary
in relation to making the basic Gillespie method more efficient. An
excellent and highly readable review of the current state of
numerical methods is given by~\citep{Gillespie:2007}. Here we will
briefly review some of the software available for stochastic
simulation, we will not consider hybrid methods which combine
discrete with continuous modeling since the methods are still
under development (see~\citep{Salis:2006} for details), however a
brief review is given in~\citep{Ullah:2007}. We will also not
consider here spatial-stochastic modeling tools such as
MCell~\citep{MCell:2001}.

Unlike deterministic models where a single run provides all the
information on the trajectory, the Gillespie method only generates
one of a very large number of possible realizations. Ideally
stochastic simulators should support ensemble simulations so that
statistics can be collected on the behavior of the model. Many
simulators do not provide for this facility and presumably expect
the user to manually make the necessary repeats (which ideally
should run into tens of thousands of runs). Only a few simulators
support ensemble runs (e.g.\ COPASI) but only one software
tool~\citep
{RoaSauro:2007} computes a wide range of statistics for
analyzing stochastic models.

As with differential equation based models, there is a test suite
available that allows developers of stochastic simulators to
evaluate their efforts (See section 6). The test suite was developed
and is maintained by the Wilkinson group at Newcastle,
England~\citep{Evans:2007}. An example of how the test suite can be
used is given in~\citet{RoaSauro:2007}.

As for software, there are a number of tools that support stochastic
simulation, many of these include the tools we have described in
previous sections, for example, COPASI and Jarnac. SBW also
incorporates a number of stochastic solvers, including a high speed
solver written using .NET. Other more specific tools that focus
exclusively on stochastic simulation include
BioNetS~\citep{Adalsteinsson:2004} which is a Mac based tool (a
version with a Java frontend also exists for Windows and Linux) that
incorporates a number of stochastic related solvers including
Gillespie and a capability to solve Langevin equations. This later
ability is rare and makes BioNetS a very useful tool for this kind
of work.

Two other well known applications, include Dizzy and StochSim.
Dizzy~\citep{Ramsey:2005} is a Java based  application that was
developed by Stephen Ramsey at the Institute of Systems Biology in
Seattle and incorporates a number of stochastic methods. It is
capable of reading SBML level 1 and has an SBW programmatic
interface. StochSim~\citep{LeNovere:2001} was developed in the 1990's at
Cambridge University, UK by Carl Firth in Bray's group and subsequently developed further by Le~Nov{\`e}re and Shimizu. It uses a slightly different method to the Gillespie approach which according
to the authors scales better when a system contains molecules that
can exist in multiple states. In addition later versions have
provision for spatial simulation in two dimensions. One significant
difference is that StochSim uses a fixed time step whose size will
determine the accuracy of the algorithm. In contrast, the Gillespie
SSA method uses an exact approach where the step size of computed as
part of the algorithm itself. Subsequent variants of SSA such as
tau-leaping are also however approximate and for detailed
statistical analysis of stochastic run, such approximate methods
should probably be avoided.

Other examples of stochastic simulators include
Cyto-Sim~\citep{Sedwards:2007} from the Microsoft Trento group
written in Java and ESS from the University of
Tennessee~\citep{cox:2003}. Cyto-Sim incorporates an interesting and
easy to learn script language for describing biochemical networks.
ESS and its descendants are notable for implementing probably the
worlds fastest stochastic simulation software.

Finally, there currently exists one software library, StochKit, that
is dedicated to implementing a wide variety of Gillespie based
approaches~\citep{StochKit:2007} and is being being developed by the Petzold group in Santa Barbara, USA. StochKit is a C++ based library and therefore can in principle only be linked easily to other C++ applications. A mundane but more useful C based API (Application Programming Interface) would make the library much more portable to other software languages. In addition StochKit currently lacks many of the advanced analysis methods that a stochastic solver would need for practical use. Finally, support for Windows appears to be very limited which makes the library essentially a Linux based tool. StochKit is a promising start but has someway to go before it reaches the usability and portability of libraries such as SUNDIALS~\citep{CVODE:2005} or ODEPACK~\cite{HIND83}. The Tennessee group that developed ESS~\citep{cox:2003} is developing a very high performance C based solution that will incorporate many advanced features such as support for discrete events and ensemble simulations. In addition there are dedicated hardware solutions to stochastic simulation which involve the use of field programmable gate arrays and other technology to build what could potentially be very high speed simulation engines~\citep{peterson:2002,mccollum:2003,yoshimi2004}.


\section{Standards}

With the surge in the number of incompatible simulation tools since
the year 2000, it was realized by at least two communities that some
form of standardization for model exchange was necessary. The two
standards that emerged, include CellML and SBML. CellML is primarily
a notation for representing biochemical models in a strict
mathematical form, as a result it is in principle completely
general. SBML on the other hand uses a biologically inspired
notation to represent networks from which a mathematical model can
be generated. Each has its strengths and weaknesses, SBML has a
simpler structure compared to CellML, as a result there is more
software support for SBML. Most software tools at the present time
support import and export of SBML. Both standards have very active
communities with intra cellular models being primarily the domain of
SBML and physiological models for CellML.

\subsection{CellML}

CellML~\citep{hedley:2001b,LloydCellML2004} represents cellular models using a
mathematical description. In addition, CellML represents entities
using a component based approach where relationships between
components are represented by connections. The literal translation
of the mathematics however goes much further, in fact the
representation that CellML uses is very reminiscent of the way an
engineer might wire up an analog computer to solve the equations
(though without specifying the integrators). As a result CellML is
very general and in principal could probably represent any system
that has a mathematical description. CellML is also very precise in
that every item in a model is defined explicitly. However, the
generality and explicit nature of CellML also results in increased
complexity especially for software developers.

Another key aspect of CellML is its provision for metadata support.
The metadata can be used to provide a context for a model, such as
the author name, when it was created and what additional documents
are available for it's description. CellML uses standard XML based
metadata containers such as RDF and within RDF the Dublin Core.
CellML metadata, such as BioPAX
(\url{http://www.biopax.org},~\citet{BioPax:2007}), is how biological
information can be introduced into a CellML model.

The CellML team has amassed a very large suite (hundreds) of
models(\url{www.cellml.org/models/},~\citet{Lloyd:2008}) which
provides many real examples of CellML syntax. This is an extremely
useful resource for the community.

\subsection{SBML}

Whereas CellML attempts to be highly comprehensive, SBML was
designed to meet the immediate needs of the modeling community and
is therefore more focused on a particular problem set. One result of
this is that the standard is simpler compared to CellML although
more recent revisions add new functionality so that the difference
in complexity between CellML and SBML is becoming less significant.
Like CellML, SBML is based on XML, however unlike CellML, it takes a
different approach to representing cellular models. The way SBML
represents models, closely maps the way existing modeling packages
represent models. Whereas CellML represents models mathematically,
SBML represent models as a list of chemical transformations. Since
every process in a biological cell can ultimately be broken down
into one or more chemical transformations this was a natural
representation to use. However SBML does not have generalized
elements such as components and connections, SBML employs specific
elements to represent spatial compartments, molecular species and
chemical transformations. In addition to these, SBML also has
provision for rules which can be used to represent constraints,
derived values and general math which for one reason or another
cannot be transformed into a chemical scheme.

SBML, like any standard, evolves with time~\citep{Finney2003}. Major
revisions of the standard are captured in levels, while minor
modifications and clarifications are captured in versions. An
example of a major change within the standard would be the use of
MathML in level two of SBML, whereas level one encoded infix strings
to denote reaction rates and rules. A minor change on the other hand
would for example be the introduction of semantic annotations that
can be added to SBML level two version three, whereas this was not
possible in a supported fashion in earlier versions (see
section~\ref{sec:SBO}). As of this writing, SBML level three is
still in development. With level three the standard will develop in
an extensible manner. This means there will be a set of core
features that must be supported around which additional features,
such as spatial modeling, can be included.

There has been little effort to develop methods to inter-convert
between SBML and CellML, however Schilstra from the UK developed a
tool called CellML2SBML~\citep{Schilstra:2006} that allows users to
convert CellML based models into SBML. It is also possible to use
tools such as VCell to inter-convert between these two standards.


\subsection{Other Related Standards}

\subsubsection{Graphical Layout}

Graphical modeling applications~\citep{BergmannCP:2006} routinely
enhance computational models by layout annotations. Recently the
SBML community has decided on a common standard on how to embed the
layout information within SBML. The layout
extension~\citep{Guages2006} allows a model to store the size and
dimension of all model elements, along with textual annotations and
reactions. Originally the intention was to embed the layout
extension in a model annotation for level 2 versions of SBML but
with the upcoming level 3 the layout extension will be added to the
SBML as a first-class construct. LibSBML has been modified to
provide access to all elements of the Layout Extension. Also several
reference implementations exist \citep{BergmannCP:2006,Deckard2007}.

Whereas the layout extension is concerned with representing simple
elements, the Systems Biology Graphical Notation (SBGN)
(\url{http://sbgn.org}) aims to standardize the visual language of
computational models unambiguously. While this standard is still in
development and strictly speaking independent of the SBML effort,
experience in other fields such as electrical engineering has
demonstrated the essential need for standardizing the visual
notation for representing models in diagrammatic form.

\subsubsection{MIRIAM}

Model Definition Languages such as SBML and CellML target the
exchange of models. They aim to pass on the quantitative
computational models from one software tool to another. However
these description formats do not concern themselves with semantic
annotations. Both SBML and CellML have launched efforts to remedy
this problem. Both communities agreed on the Minimum Information
Requested In the Annotation of biochemical Models
(MIRIAM,~\citet{miriam}). These annotations aim to further the
confidence in quantitative biochemical models, making it easier and
more precise to search for particular biochemical models, enabling
researchers to identify biological phenomena captured by a
biochemical model and perhaps most importantly to facilitate model
reuse and model composition.

In order to call a model MIRIAM compliant, the model has to be encoded in a standard format, such as SBML. Furthermore it needs to be tied to a reference description, describing the properties and results that can be obtained from the model. Parameters of the computational model have to be provided so that the model can be loaded into a simulation environment where the results can be reproduced. Other information that has to be provided is a name for the model, the creator of the model the date and time of the last modification as well as a statement about the terms of distribution.


\subsubsection{SBO -- Systems Biology Ontology}\label{sec:SBO}

In order to assign meaning to model constituents an ontology
specific to Systems Biology has been developed: The Systems Biology
Ontology (SBO, \url{http://www.ebi.ac.uk/sbo/}). The ontology consists of five controlled vocabularies and two relationships: is-part-of and is-a.
Qualifying model participants, say as enzyme, macromolecule,
metabolite or small species such as an ion will make it easier to
generate meaning from the model. It will make the generation of
standard visual notations such as SBGN possible. Moreover it
presents a solution on how to interpret the model computationally,
as the SBO allows tagging a model as continuous, discrete or logical
model. One could even go a step further, making kinetic interactions
in a model obsolete, by just referencing that the rate law is one
specified by an ontology identifier (e.g.: tagging a reaction as
following Henri-Michaelis Menten enzyme kinetics and specifying the
parameters). The SBO is community driven and new terms or
modifications to the existing onotology can be requested by the
community.

\subsection{Other Ontologies}

The most recent developments in CellML and particularly the SBML
communities revolve around the creation of ontologies and refining
the exchange semantics. Apart from classifying model constituents
with an appropriate ontology, one of the current areas of interest
is describing the dynamical behavior of a model. The ``Terminology
for the Description of Dynamics`` (TEDDY,
\url{http://www.ebi.ac.uk/compneur-srv/teddy/}) provides a rich
ontology to describe and quantify what kind of behavior a
computational model is able to exhibit (e.g.: the characteristics of
a model could describe bifurcation behavior where the functionality
of a model could be described as featuring oscillations or switch
behavior). However knowing that a model exhibits interesting
behavior is not enough: more information is needed in order to
recreate that behavior. The ``Minimum Information About a Simulation
Experiment'' (MIASE, \url{http://www.ebi.ac.uk/compneur-srv/miase/}) project focuses in
this problem. MIASE will help to describe the simulation algorithms
and the simulation tool used along with all needed parameter
settings. In order to do so it will use the Kinetic Algorithm
Ontology (KiSAO) that relates simulation algorithms and methods to
each other. As these ontologies are still currently under
devlopment, it will be interesting to see how they progress and are
taken up by the community.

Lastly we should mention BioPAX~\citep{BioPax:2007}, Biological Pathway Exchange. BioPAX is an XML based format that will act as a bridge between different
pathway databases and data. In relation to modeling software, BioPAX
may offer a means to embed rich annotation data into an SBML or
CellML model. Some of this capability is being addressed to a
limited extend by the new ontologies being developed at EBI in
Cambridge, UK. However, BioPAX, given it role to allow a common
exchange of biological data between pathway database, may offer a
useful complementary way to bind data to computational models.

\subsection{Human Readable Formats}

SBML and CellML are examples of formats that use XML to represent
information. One advantage to using XML is that there is much
software available to assist in reading and manipulating XML based
data. However, XML is not suited for human consumption but is
designed strictly to be read by computer software. In order for
humans to build and read models, human readable formats are
required, often these are text based but sometimes they are graphical. In relation to text based formats, there has been a long tradition to using human readable formats for representing
biochemical models, starting with BIOSSIM~\citep{Garfinkel:1970}.
Other examples of early human readable formats include work by
Park~\citep{PW73} and Burns~\citep{Bu71} to cite but a few. In more
recent years simulators such as SCAMP~\citep{SauroF91} and
METAMOD~\citep{HM86}  also introduced human readable formats to
define models. Both software tools were developed in later years into Jarnac and PySCeS respectively.

Other formats of interest include composable languages developed by
James McCollum at the University of Miami, Sauro and
Bergmann~\citep{BergmannShortHand} at the University of Washington
and Michael Pederson at the University of
Edinburgh~\citep{Pederson:LBS:2008}. Blinov, Faeder, Goldstein and
Hlavacek developed BioNetGen~\citep{Blinov:2004} which is a rule
based format for representing systems with multiple states,
Cyto-Sim, which we have mentioned previously, incorporates an
interesting human readable language for representing biochemical
systems. The SBML community~\citep{SBMLShortHand} has also developed
a human readable script called SBML-shorthand. This notation maps
directly on to SBML but is much easier to hand write compared to
SBML (as are all these human readable formats). The shorthand is
also much less verbose and uses infix to represent expressions
rather than MathML. Finally we should mention a lisp based language
called little b~(\url{http://www.littleb.org/}) being developed at
Harvard University. The aim of little b is to allow biologists to
build models quickly and easily from shared parts.

\section{Databases}

Along with the standardization of model representation there has
been an obvious desire to create model repositories where models
published in journals can be stored and retrieved. There are at the
present time, five repositories with varying degrees of quality and
usability. Probably the most promising is the UK based searchable ,
BioModels Database, which at the current time (July 2008) holds over
one hundred and seventy fully curated and working models that can be
downloaded in standard SBML as well as other formats. BioModels also
has the great benefit of providing programmatic access to its
database via web services which allows any software program to
access the database seamlessly across the internet. Models stored in
the BioModels Database are curated, meaning that models will
reproduce the original's authors intention. In addition, the models
are liberally annotated so that model components can be referenced
from other database sources.

Another large database has been assembled by the CellML community~\cite{Lloyd:2008}
which has over three hundred models stored in CellML format. From their site it is possible in principle to convert the CellML into standard C code for compilation in to a working model.

The JSim group at the University of Washington has a large database of physiological models~\url{http://nsr.bioeng.washington.edu/Models/} stored in the mathematical language used by the JSim simulation application. These models can only be read by JSim and currently there is no simply way to translate these to any of the common exchange formats though this is likely to change in the future.

Another small but very useful database is the JWS online database
developed by Brett Olivier and Jacky Snoep~\citep{olivier:2004} which
has almost eighty fully working models. JWS allows export in both
SBML and the script format PySCeS which can be easily translated to
other formats such as Jarnac script. JWS is arguably one of the
first databases of models although physiological models such as
those supported by JSim have been available for longer. Many if not
all the models on JWS have also been ported to the BioModels
Database and vice versa.

Another database, DOQCS~\url{http://doqcs.ncbs.res.in/}focuses on signaling networks which contains
over two hundred models. Models in DOQCS can only be downloaded
however in Genesis format~\citep{kinetikit:2002} which limits the portability to other
frameworks. Recently the DOQCS database has been merged with the
BioModels Database.

Finally there is a major NIH sponsored database called
Sigmoid~\citep{Sigmoid:2005} which currently has about twenty models.
The focus of the Sigmoid project however appears to be
infrastructure rather than curation which explains the limited
number of available models. Access to the database is limited to the
model explorer tool SME (\url{www.sigmoid.org}), thus access from other applications may not
be possible. In addition models can only be accessed from the web
interface using Cellerator format which limits portability to other
frameworks. Work on Sigmoid is ongoing and no doubt changes will
occur in the future to make it more open.

\section{Test Suites}

An important need in the simulation community is some means to
compare and test simulation codes. Currently there are available a
number of test suites expressed in SBML. One is provided by Andrew
Finney~(\url{http://sbml.org/wiki/Semantic_Test_Suite}) which
provides various tests for deterministic models and another by Evans
Evans, Gillespie and Wilkinson.,~\citep{Evans:2007} which represents a stochastic test suite.
Currently the deterministic test suite is undergoing a redesign
because it has significant issues related to numerical stability.
The stochastic test suite has recently gone through a second
iteration and is an extremely useful resource. Another suite of
models that can be used for testing is the BioModels Database of
models. Although test results are not available, it is possible to
use the database models for comparison purposes as has been done by
Bergmann~(\url{http://www.sys-bio.org/sbwWiki/compare},\citep{bergmann:2008}).
The comparison site gives detailed side-by-side information on how
different simulators solve a given SBML model. Overall agreement of
the simulation packages appears to be high.

The CellML team in Auckland are in the process of developing a test
suite for CellML which should aid considerably in testing CellML
compliant applications.


\section{Future Prospects}

In surveying the development of software in systems biology we see a
vibrant and sometimes innovative community with a very wide range of
tools to satisfy all manner of users. There are still some areas
that are lacking, most notably bifurcation analysis and model
composition. There are some very notable tools for bifurcation
analysis but they have been written more for general use than
specifically for systems biology (Oscill8 being a notable
exception). User interface design is still somewhat primitive in
these bifurcation tools and there is much opportunity for innovation
in this area particularly with respect to interactive bifurcation
analysis. The second area that is lacking is model composition ({\em
cf.} ProMoT). As models become more common place, there is a growing
desire to be able to take models and combine them easily. Currently
this is not possible without considerable effort. Model composition
is not a simple problem to solve however as there are many issues to
consider including unit consistency and interface protocols. The
benefits would be significant however, and with the rise of
synthetic biology there is even more reason to develop the idea.

\subsection{Reusable Software Libraries}

One of the issues that has plagued software development in systems
biology is chronic reinvention. Many tools, particularly the tools
listed at SBML.org carry out similar functions at their core
(solving ODEs, computing steady states, sensitivities etc). The fact
that each tool reimplements the same functionality is arguably a
waste of resources and it would be of great benefit to the community
if some of the core functionality could be released in the form of
reusable software libraries. There are many benefits to such an
approach including, improved testing, documentation,
maintainability, and extensibility. Unless there are strong reasons
to do so, very few developers now write their own stiff ODE solver
or SBML reader. There are already some useful libraries in the community such as
libSBML, SUNDIALS, ODEPACK, LAPACK etc. These libraries are
successful for a number of reasons, they are well documented, open,
and most importantly, they are written in languages (C and FORTRAN)
that make it easy to interface them to other programming languages.
Libraries that are written in other languages, such as Java or C++,
or where a C interface is not exposed, tend to be less successful.
We have seen some efforts to develop additional libraries, most
notably SOSLib and StochKit, but these are still under development
and lack some critical functionality (SOSLib lacks the ability to
remove dependent species) or ease of integration in the case of
StochKit. These issues will no doubt be resolved in the future and
will then allow developers to focus on other areas of interest such
as user interface design and the development of new analysis
approaches.

\section*{Acknowledgements}

We would like to acknowledge the generous support from a number of
funding agencies including the US Department of Energy GTL program,
and the NIH (1R01GM081070-01). We would also like to acknowledge the
many useful discussions we have had over the years with our colleagues in the
software and computational systems biology community, in particular we would like to acknowledge Athel Cornish-Bowden, David Fell, Jannie Hofmeyr, Mike Hucka and Pedro Mendes,

\bibliography{SauroBib}
%\bibliographystyle{plainnat}
%\bibliographystyle{acm}
\bibliographystyle{apa}


\end{document}
